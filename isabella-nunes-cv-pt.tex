\documentclass[10pt,a4paper,ragged2e,withhyper]{assets/class/altacv}

\geometry{left=1.25cm,right=1.25cm,top=1.5cm,bottom=1.5cm,columnsep=1.2cm}

\usepackage{paracol}

\ifxetexorluatex
  \setmainfont{Roboto Slab}
  \setsansfont{Lato}
  \renewcommand{\familydefault}{\sfdefault}
\else
  \usepackage[rm]{roboto}
  \usepackage[defaultsans]{lato}
  \renewcommand{\familydefault}{\sfdefault}
\fi

\definecolor{SlateGrey}{HTML}{2E2E2E}
\definecolor{LightGrey}{HTML}{666666}
\definecolor{DarkPastelRed}{HTML}{450808}
\definecolor{PastelRed}{HTML}{8F0D0D}
\definecolor{GoldenEarth}{HTML}{E7D192}
\colorlet{name}{black}
\colorlet{tagline}{PastelRed}
\colorlet{heading}{DarkPastelRed}
\colorlet{headingrule}{GoldenEarth}
\colorlet{subheading}{PastelRed}
\colorlet{accent}{PastelRed}
\colorlet{emphasis}{SlateGrey}
\colorlet{body}{LightGrey}

\renewcommand{\namefont}{\Huge\rmfamily\bfseries}
\renewcommand{\personalinfofont}{\footnotesize}
\renewcommand{\cvsectionfont}{\LARGE\rmfamily\bfseries}
\renewcommand{\cvsubsectionfont}{\large\bfseries}

\renewcommand{\itemmarker}{{\small\textbullet}}
\renewcommand{\ratingmarker}{\faCircle}

\begin{document}
\name{Isabella de Freitas Nunes}
\tagline{Desenvolvedora de Software}
\photoR{2.8cm}{assets/img/profile}

\personalinfo{
  \email{isadfrn@gmail.com}
  \phone{+55 64 99979 9732}
  \location{Blumenau, SC, Brazil}
  \homepage{www.isadfrn.dev}
  \twitter{isadfrn}
  \linkedin{isadfrn}
  \github{isadfrn}
}

\makecvheader

\columnratio{0.6}

\begin{paracol}{2}

  \cvsection{Sobre}

  Desenvolvedora de software que gosta de Algoritmos, Estruturas de Dados, Linux, projetos Open Source e ensino de código. Trabalhou em grandes projetos e interagiu com pessoas de muitos lugares. Gosta de aprender, fazer parte de projetos desafiadores, lidar com tecnologias e pessoas. Além disso, ensina código para iniciantes que desejam aprender a programar há mais de sete anos. Experiência em desenvolvimento Full-Stack, construindo uma carreira com abordagem em forma de T com profundidade no desenvolvimento back-end. No momento em transição de carreira para atuar com desenvolvimento para dispositivos móveis.

  \medskip

  \cvsection{Experiência}

  \cvevent{Desenvolvedora de Software}{Zup Innovation}{Dez 2021 -- Apr 2022}{Uberlândia, MG, Brasil}
  Principais atividades:
  \begin{itemize}
    \item Atuação remota em projetos de agronegócios do Banco Itaú;
    \item Time ágil trabalhando com Kanban, usando o Jira para gestão.
    \item Stack e ferramentas: Java, Spring Framework, JavaScript, TypeScript, Node.js, JUnit, Mockito, Docker, Kafka, Api Gateway, EC2 e DynamoDB;
  \end{itemize}

  \divider

  \cvevent{Desenvolvedora de Software}{Philips}{Dez 2019 -- Jun 2021}{Blumenau, SC, Brasil}
  Principais atividades:
  \begin{itemize}
    \item Atuação remota em projetos de healthcare, interagindo com times dos EUA, India e Países Baixos;
    \item Time ágil trabalhando com Scaled Agile Framework (SAFe), usando o Azure DevOps para gestão.
    \item Stack e ferramentas: Java, Kotlin, Spring Framework, PL/SQL, Oracle Database, PostgreSQL, MongoDB, JUnit, Mockito, Docker, Kafka, Jenkins, GitHub Actions, Sonar, JavaScript, HTML, CSS, React.js, Angular e SharePoint;
    \item Sistemas: Philips Tasy;
  \end{itemize}

  \divider

  \cvevent{Instrutora de Programação}{Blue EdTech}{Mai 2021 -- Jun 2022}{São Paulo, SP, Brasil}
  Principais atividades:
  \begin{itemize}
    \item Atuação remota em projetos do setor educacional;
    \item Stack e ferramentas: JavaScript, TypeScript, Node.js, Express.js, HTML, CSS, React.js, NestJS, Next.js, Prisma, Mongoose, PostgreSQL, MongoDB e Docker;
  \end{itemize}

  \cvevent{Desenvolvedora de Software}{Nutrien Soluções Agrícolas}{Jul 2020 -- Abr 2021}{Rio Verde, GO, Brasil}
  Principais atividades:
  \begin{itemize}
    \item Atuação remota em projetos de agronegócios;
    \item Time ágil trabalhando com Kanban, usando o Jira para gestão.
    \item Stack e ferramentas: Java, JavaScript, Python, PL/SQL, AdvPL, Django, Node.js, React.js, React Native, PyQt, MongoDB, MySQL, PostgreSQL, Oracle Database, Docker, Portainer e Kubernetes;
    \item Sistemas: Totvs Protheus, Senior, Simple Sync, Simple Field, Simple Sales e Fretefy;
  \end{itemize}

  \divider

  \cvevent{Instrutora de Programação}{Senac}{Mai 2015 -- Dez 2020}{Jataí, GO, Brasil}
  Principais atividades:
  \begin{itemize}
    \item Atuação presencial em projetos do setor educacional;
    \item Stack e ferramentas: Java, Visual Basic for Applications, HTML, CSS, JavaScript e Microsoft Excel;
  \end{itemize}

  \divider

  \cvevent{Desenvolvedora de Software}{J. Cruzeiro}{Jul 2018 -- Jul 2020}{Jataí, GO, Brasil}
  Principais atividades:
  \begin{itemize}
    \item Atuação presencial em projetos do segmento da construção civil;
    \item Stack e ferramentas: Java, Spring Framework, PL/SQL, Oracle Database, JavaScript, HTML, CSS e React.js;
    \item Sistemas: ADM (Santri Sistemas), Senior, Sienge, Linx Degust, Zabbix, Windows Server, Active Directory, Samba, Citrix XenServer, XenApp, XenDesktop, Director, Netscaler, pfSense, Sophos XG e Openfire Spark;
    \item Hardware: UTM Sophos, IBM Flex System x240, IBM DS4200, IBM V3700, IBM Tivoli e IBM TSM3100;
  \end{itemize}

  \medskip

  \cvsection{Um dia na minha vida}

  \wheelchart{1.5cm}{0.5cm}{
    6/8em/accent!30/Descanso,
    3/8em/accent!40/{Series, Filmes, Jogos e Músicas},
    8/8em/accent!60/Trabalho,
    2/10em/accent/Esportes,
    5/6em/accent!20/Tempo com a família
  }

  \newpage

  \switchcolumn

  \cvsection{Filosofia de vida}

  \begin{quote}
    ``Se você não consegue fazer, pratique até conseguir. Se você consegue fazer, pratique até atingir a perfeição. Se você atingiu a perfeição uma vez, pratique até atingir a perfeição em todas as tentativas.``
  \end{quote}

  \medskip

  \cvsection{Conquistas}

  \cvachievement{\faTrophy}{Monitoria Acadêmica}{Na graduação fui monitora de Algoritmos II, Estruturas de Dados I, Estruturas de Dados II, Engenheira de Software, Projeto de Software e Arquitetura de Computadores}

  \medskip

  \cvsection{Habilidades}

  \cvtag{Hard-working}
  \cvtag{Detalhista}
  \cvtag{Resiliente}
  \cvtag{Autodidata}
  \cvtag{Comunicação}

  \divider\smallskip

  \cvtag{JavaScript}
  \cvtag{Java}
  \cvtag{Python}
  \cvtag{React}
  \cvtag{Next.js}
  \cvtag{Node.js}
  \cvtag{Express}
  \cvtag{Prisma}
  \cvtag{Spring}
  \cvtag{NestJS}
  \cvtag{MongoDB}
  \cvtag{PostgreSQL}
  \cvtag{Oracle database}
  \cvtag{HTML}
  \cvtag{CSS}
  \cvtag{Git}
  \cvtag{GitHub}
  \cvtag{Docker}

  \medskip

  \cvsection{Línguas}

  Português \hfill Nativo

  \divider

  Inglês \hfill C1

  \medskip

  \cvsection{Educação}

  \cvevent{Pós-graduação em Java}{Universidade Tecnológica Federal do Paraná}{Mar 2022 -- Jul 2023}{}

  \divider

  \cvevent{Bacharela em Ciências da Computação}{Universidade Federal de Goiás}{Fev 2014 -- Dez 2018}{}

  \newpage

  \cvsection{Cursos}

  \cvevent{\href{https://github.com/isadfrn/curriculum/blob/master/certificates/2022/udemy-java-complete.pdf}{Java + Orientação a Objetos + Projetos (51 h)}}{Udemy}{Mar 2020}{}

  \divider

  \cvevent{\href{https://github.com/isadfrn/curriculum/blob/master/certificates/2021/startechfullstack.pdf}{Start Tech - Full Stack - JavaScript, Node e React (120 h)}}{Gama Academy}{Ago 2021}{}

  \divider

  \cvevent{\href{https://github.com/isadfrn/curriculum/blob/master/certificates/2021/wecancodeacademy.pdf}{We Can Code Academy - C sharp (120 h)}}{Gama Academy}{Fev 2021}{}

  \divider

  \cvevent{\href{https://github.com/isadfrn/curriculum/blob/master/certificates/2020/gostack2020.pdf}{Bootcamp GoStack 13 - JavaScript, Node e React (160 h)}}{Rocketseat}{Ago 2020 -- Dez 2020}{}

  \divider

  \cvevent{\href{https://github.com/isadfrn/curriculum/blob/master/certificates/2020/gostack2020.pdf}{Bootcamp Philips Fullstack Developer (132 h)}}{Digital Innovation One}{Mar 2022 -- Abr 2022}{}

\end{paracol}

\end{document}
